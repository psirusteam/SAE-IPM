\documentclass[a4paper, 11pt]{article}
\usepackage{apacite}
\usepackage{natbib}
\bibliographystyle{apalike}
\usepackage[dvipsnames]{xcolor}
\usepackage{amsmath}
\usepackage{amsmath}
\usepackage{subfigure}
\usepackage{graphicx}
\usepackage{url,booktabs,multirow}
\usepackage{hhline}
\usepackage{amsmath,amsfonts,amssymb,amsbsy,amsthm,bm}
\usepackage{nicefrac}
\usepackage[normalem]{ulem}
\usepackage{subfigure}
\usepackage{color}
\usepackage{times}
\usepackage{setspace}
\usepackage[bottom]{footmisc}
\usepackage{caption}
\captionsetup{font=small}
\usepackage[flushleft]{threeparttable}
\graphicspath{ {./Figures/} }
\usepackage{lineno,hyperref,amsmath}
\renewcommand{\labelenumii}{\theenumii}
\renewcommand{\theenumii}{\theenumi.\arabic{enumii}.}
%\usepackage{draftwatermark}
\usepackage{xparse}

\usepackage[T1]{fontenc}
\usepackage[utf8]{inputenc}
\usepackage{authblk}
\NewDocumentCommand{\tens}{t_}
 {%
  \IfBooleanTF{#1}
   {\tensop}
   {\otimes}%
 }
\NewDocumentCommand{\tensop}{m}
 {%
  \mathbin{\mathop{\otimes}\displaylimits_{#1}}%
 }
 
\title{Multiple deprivation index using small area estimation methods: an application for the adult population in Colombia}


\author[1]{Andrés Gutiérrez-Rojas}
\author[2]{Alejandra Arias-Salazar}
\author[1]{Diego Lemus-Polanía}
\author[3]{Natalia Rojas-Perilla}
\author[1]{Xavier Mancero}
\affil[1]{Economic Commission for Latin America and the Caribbean}
\affil[2]{Freie Universität Berlin}
\affil[3]{United Arab Emirates University}

%\affil[1]{Statistics Division, Economic Commission for Latin America and the Caribbean}
%\affil[2]{Department of Economics, Freie Universität Berlin}
%\affil[3]{Department of Analytics in the Digital Era, United Arab Emirates University}


\renewcommand\Authands{ and }


\date{}
\begin{document}
\maketitle


\begin{abstract}\small

Based on the new ``Multiple deprivation index" (MDI) for Latin American countries produced by the Economic Commission for Latin America and the Caribbean, this paper shows the case study of Colombia to obtain estimates of this multidimensional index in small domains. This country has a recent population census (2018) providing most of the information required to compute the MDI at municipality level. However,  two of the indicators required to compute the index are not available in the census, namely: ``Employment-related income" and ``Lack of social security". Therefore, small area estimation (SAE) methods are implemented to obtain unit-level estimates for those specific indicators. Since two different SAE procedures are carried out, a parametric bootstrap algorithm is proposed to provide respective uncertainty measures.

\end{abstract}

{{\bf \noindent Keywords}: Small area estimation; Multidimensional poverty; Composite indicators; Empirical best predictor; Generalized linear mixed model}	

\clearpage

\section{Introduction}


Poverty is, and have been, one of the leading topics in national and international agendas for decades. A recent example is the first goal of the 2030 agenda for Sustainable Development (SDG): ``End poverty in all its forms everywhere", as well as its 1.2.2 indicator which measures ``proportion of men, women and children of all ages living in poverty in all its dimensions according to national definitions" \citep{UnitedNationsGeneralAssembly2015ResDevelopment}. In this sense,  obtaining quality data on poverty in its different expressions takes relevance as well as produce desegregated information (e.g. geographically or considering characteristics of the population) to identify and develop strategies to eradicate it.  

Traditionally national and international organisms require the uni-dimensional poverty measure based on income and/or expenditure. The well known Foster-Greer-Thorbecke (FGT) poverty measures \citep{foster1984}, provide information about the head count ratio, poverty gap index and the severity of poverty. However, several studies had revealed that poverty is a complex phenomenon that should be analyzed considering a group of factors and not only one, usually monetary (Lemmi, A. A., and Betti, G. (Eds.), 2006; Belhadj, B., & Limam, M., 2012). Alkire & Foster (2007), proposed a methodology based on previous research to measuring poverty, taking into account not achievements, but deprivations of individuals and households. Recently, this methodology has been widely used by many countries because there is no restriction on the number of dimensions and indicators, and furthermore, each nation can define the composite of their index.
In the case of Latin America, it has been clarified that  considering not only the monetary approach but also the multidimensional one, has helped to identify that poverty is a structural phenomenon in the region \citep{CEPAL2014}. Due to the complexity of the problem, alternatives have been sought in the region to have a standardized methodology that allows not only to study the phenomenon in an adequate way, but also to be able to compare between countries. In the early eighties the Economic Commission for Latin America and the Caribbean (ECLAC), presented the unsatisfied basic needs (UBNS) index as a method to represent the non-monetary capacities of the persons and households, but at the same time highly correlated with income \citep{Mancero2001}

\textcolor{red}{More recently, in 2022 the multidimensional deprivation index (Cepal) has the purpose of ..... in comparison with the UBN.} 

As aforementioned, desegregation is also required. The first alternative are administrative records and population and housing censuses because they collects information at individual level. In Latin America countries lack of administrative systems and quality usually does not fulfil necessities to produce poverty information. On the other hand, census are produced every 5- 10 years, income information is not available and other relevant information is also not there. The other alternative, is to use survey data having the limitation of small sample sizes. 

Small area estimation (SAE) methods have the goal of producing reliable estimates in smaller domains, i.e. with adequate precision, by combining two or more sources of information. Most of these methodologies, usually classified as unit- or area-level models, provide efficiency gains if the correlation between existing auxiliary information and the survey data is sufficient (\citealt{Pfeffermann2013, Rao2015,  Pratesi2016, tzavidis2018start}).  Since the idea is to estimate the MDI for small domains, but also without losing the information of each indicator and dimension, unit level models are chosen. 


The objective of this paper is to tackle the multidimensional problem of poverty estimation in Colombia by providing an estimation of the multidimensional deprivation index (MDI). To provide estimates of the MDI in small domains in Colombia, model-based estimation methods are applied and combined. Those methods are based on SAE methodologies that are data-driven and take the theoretical information from the survey into account. For this, the pseudo empirical best predictor (EBP) is applied in parallel to an empirical plug-in methodology to provide different indicators dimensions and the global MPIs for small domains. As uncertainty measure, the Mean Squared Errors are estimated via parametric bootstrap.
The structure of the paper is as follows: Data sources and relevant characteristics of this country are mentioned in Section 2. SAE methods included in this study are briefly explain in Section 3, as well as the procedure to obtain MSE estimates for the corresponding point estimates. In section 4 we present a simulation exercise to validate our proposal. Section 5 is dedicated to show main results. Conclusions and further research are presented in Section 6. 


%\textbf{Different definitions of poverty:} 
 %There are different approaches of studying poverty with the aim of implementing the necessary strategies to reduce it through public policies and social programs. Wagle (2008) identified three: the economic well-being, the social inclusion, and the capability approach which is based on Sen(1993). Hagenaars and de Vos (1988) identified the absolute, relative and self-assessed poverty, where the latter has a subjective connotation \citep{weziak2014}. 

%\textbf{Different ways of measure poverty:} 


%a) Unidimensional methods:

%-The well known Foster-Greer-Thorbecke (FGT) poverty measures \citep{foster1984}, provide information about the head count ratio, poverty gap index and the severity of poverty. 

%-Laeken indicators




%-Fuzzy Monetary Depth Indicators (is it multidimensional?)
    


%\textbf{Colombian case study:}

%-   Explain the necessity of multidimensional poverty estimation in Colombia. 

%-   Explain why the use of SAE in this context.
%-   Explain the dimensions.
%-   Explain the work dimension.
%-   Point out the missing information problem.
%-   Point out the variables type and the need of using different SAE methods

%-   We combine SAE methods

%-   

%\begin{itemize}
 %   \item Motivation of the MPI (beyond poverty income)
  %  \item Methodological and practical gaps in MPI 
  %  \item Possible solutions and alternatives
  %  \item Necessity of the MPI for small areas (granularity \& target geographies/domains)
  %  \item Challenges of applying sae methods for the MPI: 
  %  \subitem - Necessity of working with microdata 
  %  \subitem - Model-based methodologies
  %  \subitem - Time 
  %  \subitem - Different types of indicators (continuous vs discrete)
  %  \subitem - Assessment evaluation
  %  \item Objective of this paper:
  %  \subitem - To obtain small area estimates for the MPI combining pseudo EBP and plug-in method.
 %   \subitem  -Apply/propose MSE estimation.
 %   \item Sections
%\end{itemize}



\section{Case study: multidimensional deprivation index for adult population in Colombia}

Colombia is chosen as the first country in Latin America where the MDI for small areas is implemented. Unlike other countries in the region, Colombia counts on a recent population and housing census with most all the variables required to produce the MDI. 



\subsection{Data sources}

\subsubsection{Censo Nacional de Población y Vivienda (CNPV)  2018}
Implementation of the SAE requires at least two sources of data. For this case study the first one is the population and housing census Censo Nacional de Población y Vivienda (CNPV) 2018 of Colombia. It is conducted by the National Statistical Office of Colombia (DANE: Departamento Administrativo Nacional de Estadística). Although it is planned to be every 10 years, the last census was carried out three years later as planned due to administrative and economic reasons. For first time, the information was collected via electronic census and the traditional face-to-face interview. Data collection phase was conducted during the 2018 in 10 months \citep{DANE2019}.


\subsubsection{Gran Encuesta Integrada de Hogares (GEIH) 2018}

Comprehensive survey of households (in spanish: Gran Encuesta Integrada de Hogares - GEIH) collects information on employment, income and expenses of individuals and households. This survey is part of the Household Survey Data Bank (BADEHOG), a repository of household surveys from 18 Latin American countries maintained by the ECLAC Statistics Division.  The GEIH provides results yearly, that are representative for the national, national urban, national rural, regional, departmental, and for the capitals of the country's departments.  
    

%In this case, data are obtained from the Household Survey Data Bank (BADEHOG), a repository of household surveys from 18 Latin American countries maintained by the ECLAC Statistics Division. The second data source consists of the Comprehensive Survey of Households of 2018, which is representative of the national, national urban, national rural, regional, departmental, and for the capitals of the country's departments, was used together with the 2018 National Population and Housing Census.    
    

\subsection{MDI in Colombia}

The multidimensional deprivation index proposed by ECLAC is consistent with an individual well-being approach. It is focus on adult and senior population, considering a gender perspective.
%\textcolor{red}{Add briefly information about this index based on CEPAL´s publication. Why it is important to have a new version? Why only for adults? why an individual index instead of household?}

%The index is calculated for adult (18-65 years) and senior population (>65 years). \textcolor{red}{Women 18-60 and men 18-65?}

The MDI is defined as: 
\textcolor{red}{briefly describe how it is computed.}


\begin{table}[ht]
\begin{center}
\small{
	\caption{Dimensions, indicators of the MDI and data availability in the census}
	\centering
		\begin{tabular}{llcc}
		\hline
				Dimension &Indicator&Available in census &Target population  \\
				\hline
	Housing &Housing materials & Yes & Adults and seniors\\
	& Overcrowding & Yes & Adults and seniors \\
			\hline
	Basic services &  Information communication & Yes & Adults and seniors \\
	 & technologies (ICTs) & &\\
	& Drinking water & Yes & Adults and seniors \\
	& Sanitation & Yes & Adults and seniors \\
	&Energy (electricity) & Yes & Adults and seniors \\
			\hline
	Education & Unfinished education & Yes &Adults \\
	             & Illiteracy& Yes & Seniors \\
	             		\hline
	  Employment, social  & Pension & Yes & Seniors \\
	  security and income & Employment-related income & No &Adults \\
	   & Unemployment, precarious  & No & Adults\\
	   & employment and out of the &&\\
	   &labor force due to housework&&\\
		

			\hline
	\end{tabular}
	}
	\end{center}
\end{table}

 %Target:
%\begin{itemize}
 %   \item Granularity
  %  \item Geography + cross tables
  %  \item Adults and elderly population
  %  \item Final MPI, dimensions and indicators 
  %  \item Urbanity and sex 
%\end{itemize}

%\subsection{Definition of the MPI for Latin America}
%\begin{itemize}
%\item Housing 
 %   \subitem - Housing materials
 %   \subitem - Overcrowding
%\item Basic services
%    \subitem - Information communication technologies (ICTs)
 %   \subitem - Drinking water
 %   \subitem - Sanitation
 %   \subitem - Energy (electricity)
%\item Education 
%\item Employment and social security and income
 %   \subitem - Labour force status & social security
 %   \subitem - Employment-related income
%\end{itemize}

As shown in Table 1. two of nine indicators are not available in the census and SAE methods are implemented as will be explain in next Section. 

\section{Methodology}



    The goal is to provide indicators, dimensions and the final MPI for small domains. To do that, it is necessary to use unit level models (persons).
    
    Most of the indicators are binary variables (deprivation: yes/no) and to obtain them, the EPP will be applied. However, since income information is also part of these indicators, a Pseudo EBP is also implemented. 
    
\subsection{SAE Methods}

\textcolor{red}{A definir: }
\begin{itemize}
\item \textcolor{red}{Porqué decidimos usar el EBP de Guadarrama con transformación log-shift? Cómo justificamos que esta es la mejor alternativa? Se podría justificar comparando ``clásico" EBP con EBP de Guadarrama, y/o distintas transformaciones }
\item \textcolor{red}{Lo mismo para el EPP, porqué el Plug-in en lugar del EBP para variables binarias?}
\end{itemize}

In order to performing a complete poverty mapping and obtaining a local picture of
poverty in a region, the use of an adequate estimation method becomes necessary.
Unfortunately national surveys often are not suitable to give reliable statistical information
at local levels to cover all regions within a country due to the high costs.
Small area procedures are estimation procedures for parameters under very small
sample sizes. For sufficiently large sample sizes, traditional estimators, such as the
mean estimators, produce very convincing results, but when applied to small sample
sizes, such estimators often only have very limited reliability. This is often the case
if the subject requires the data to be split into many small categories, e.g. municipalities.
Even large surveys in a country will contain administrative units from
which only very few or even no households have been sampled. For such domains, in
practice, the sampling error is often huge ([Rao03]). Luckily there is an alternative
to the classical estimators, the model based methods, which have been developed
further in recent years. These methods use model assumptions to reduce the sampling
error. The small area estimation methods, in particular, based on generalized
linear mixed models are part of this class of methods. Their basic principle is to
improve the estimation by extending the original too small sample. Explanatory a
survey data set might be extended by the census of the whole state, even though
the variable of interest is missing in the census data. Under strict assumptions on the sample and its distribution, such model based procedures result
in much better estimates than classical procedures.
    
\subsubsection{Pseudo-EBP Guadarrama (2018)}

The estimation of non-linear indicators at highly disaggregated levels is commonly carried out by the empirical best (EB) predictor introduced by Molina and Rao (2010). This method assumes a non-informative selection mechanism of individuals and can suffer from biased estimators. To address the bias problem due to a non-represntative sample under this context, the model-based pseudo empirical best (PEB) estimator, proposed by Guadarrama (2018), incorporates weights from an informative sampling design. This is a weighed EB method, which is based on the weighted sample mean and uses the inverses of the inclusion probabilities as weights.  

\textbf{NOTATION}

Let $U$ denote a finite population of size $N$ partitioned into $D$ areas or domains (representing the small areas) $U_1,U_2,\ldots,U_D$ of sizes $N_1,\ldots, N_D$, where $i=1,\ldots,D$ refers to the $i$th area. Let $y_{ij}$ be the target variable defined for the $j$th individual belonging to the $i$th area, with $j=1,\ldots,N_i$.



\textit{The FGT index of type $\alpha$ for a region $i=1,\ldots,D$ and a fixed threshold $t$ is given by}
\begin{equation}\label{eq:FGT}
F_{i}(\alpha,t)=\frac{1}{N_i}\sum_{j=1}^{N_i} F_{ij}(\alpha,t), \,\,\,\, \alpha=0,1,2
\end{equation}
\textit{where}
\begin{equation*}
F_{ij}(\alpha,t)= \left( \frac{t-y_{ij} }{t} \right)^\alpha I (y_{ij} \leq t)
\end{equation*}
with $I(A)$ an indicator function which returns 1 if $A$ is a true expression and 0 otherwise. From this definition the following poverty measures are derived:



 Denote by $\mathbf{X}=(\mathbf{x}_1,\ldots,\mathbf{x}_p)^{T}$ the design matrix containing $p$ explanatory variables and define by $s$ as the set of sample units, with $s_i$ the in-sample units in area $i$ and by $r$ be the set of non-sampled units, with $r_i$ the out-of-sample units in area $i$. Let $n_i$ denote the sample size in area $i$ with $n=\sum_{i=1}^{D}n_i$. Hence, we define by $\mathbf{y}_{i}$ a vector with population elements of the target outcome for area $i$ partitioned as $\mathbf{y}_i^T= \Big(\mathbf{y}_{is}^T,\mathbf{y}_{ir}^T\Big) $, where $\mathbf{y}_{is}$ and $\mathbf{y}_{ir}$ denote the sample elements $s$ and the out-of-sample elements $r$ in area $i$ respectively.

Consider a generic index or a target parameter $\delta_i=h(\mathbf{y}_i)$, as a function of the population variable $\mathbf{y}_i$, for $i=1,\ldots,D$. Let $\hat{\delta_i}$ an estimator of $\delta_i$ depending only on $\mathbf{y}_{is}$, the in-sample units for region $i$. The mean squared error for $\delta_i$ is defined as:
\begin{equation}\label{eq:MSE1}
\text{MSE}\left(\hat{\delta_i}\right)=\text{E}_{\mathbf{y_i}}\left\{\Big(\hat{\delta_i}-\delta_i\Big)^2\right\},
\end{equation}  
where $\text{E}_{\mathbf{y}_i}$ indicates the expectation with respect to the joint distribution of $\mathbf{y}_i$. The best predictor (EB) of $\delta_i$ that minimizes \eqref{eq:MSE1} is $\delta_i^B=\text{E}_{\mathbf{y}_{ir}}\Big(\delta_i|\mathbf{y}_{is}\Big)$, a function of $\mathbf{y}_{is}$ proportioned by the conditional expectation of $\mathbf{y}_{ir}$ given $\mathbf{y}_{is}$. Subtracting and adding $\hat{\delta_i}$ the next expression for the mean squared error is obtained as follows:
\begin{equation*}\label{eq:MSE}
\text{MSE}\left(\hat{\delta_i}\right)=\text{E}_{\mathbf{y_i}}\left\{\Big(\hat{\delta_i}-\delta_i^B\Big)^2\right\}+2\text{E}_{\mathbf{y_i}}\Big\{\left(\hat{\delta_i}-\delta_i^B\right)\Big(\delta_i^B-\delta_i\Big)\Big\}+\text{E}_{\mathbf{y_i}}\left\{\Big(\delta_i^B-\delta_i\Big)^2\right\}.
\end{equation*}
In this equation the third term does not depend on $\hat{\delta_i}$ and the second one is equal to zero as follows:
\begin{align*}
\text{E}_{\mathbf{y}_i}\left\{\left(\hat{\delta_i}-\delta_i^B\right)\Big(\delta_i^B-\delta_i\Big)\right\} &=\text{E}_{\mathbf{y}_{is}}\left[\text{E}_{\mathbf{y}_{ir}}  \left\{\left(\hat{\delta_i}-\delta_i^B\right)\Big(\delta_i^B-\delta_i\Big)\right\}|\text{E}_{\mathbf{y}_{is}}\right]\\
&=\text{E}_{\mathbf{y}_{is}}\Big[\left\{\hat{\delta_i}-\delta_i^B\right\} \Big\{\delta_i^B-\text{E}_{\mathbf{y}_{ir}}\Big(\delta_i|\mathbf{y}_{is}\Big)\Big\}\Big] \\
&=0.
\end{align*} 
Since $\delta_i^B=\text{E}_{\mathbf{y}_{ir}}\Big(\delta_i|\mathbf{y}_{is}\Big)$, it is non-negative with min-value equal to zero, the EB is:
\begin{equation}\label{eg:EXP}
\hat{\delta_i}^B=\delta_i^B=E_{\mathbf{y}_{ir}}(\delta_i|\mathbf{y}_{is}),
\end{equation}
which is also an unbiased estimator:
\begin{equation*}
\text{E}_{\mathbf{y}_{is}}\left(\hat{\delta_i}^B\right)=\text{E}_{\mathbf{y}_{is}}\Big\{\text{E}_{\mathbf{y}_{ir}}\Big(\delta_i|\mathbf{y}_{is}\Big)\Big\}=\text{E}_{\mathbf{y}_{i}}\Big(\delta_i\Big).
\end{equation*}

Usually the joint distribution of $\boldsymbol{y}$ depends on $\boldsymbol{\theta}$, a vector of unknown model parameters. In this context, the empirical best predictor (EBP) of $\delta$ can be obtained by evaluating the expectation in \ref{eg:EXP} by substituting $\boldsymbol{\theta}=\boldsymbol{\hat{\theta}}$, where $\boldsymbol{\hat{\theta}}$ is a suitable estimator of $\boldsymbol{\theta}$.   


The unit-level nested error regression model.
\begin{align}\label{eq:battese}
y_{ij}&=\mathbf{x}_{ij}^{T}\boldsymbol\beta + u_i + e_{ij}, \,\,\,\,u_i \overset{iid}{\sim} N(0,\sigma^2_u)\,\,\,\,\text{and}\,\,\,\,e_{ij} \overset{iid}{\sim}N(0,\sigma^2_e),
\end{align}
where $u_i$, the area-specific random effects, and $e_{ij}$, the unit-level error, are assumed to be independent. Assuming normality for the unit-level error and the area-specific random effects, the conditional distribution of the out-of-sample data given the sample data are also normal.  

A Monte Carlo approach is used to obtain a numerically efficient approximation to the expected value of this conditional distribution as follows:
\begin{enumerate}
	\item
	Use the sample data to obtain $\hat{{\boldsymbol{\beta}}},\hat{\sigma}_u^2,\hat{\sigma}_e^2$ and the weighting factors $\hat{\gamma_i}=\frac{\hat{\sigma}_u^2}{\hat{\sigma}_u^2 + \frac{\hat{\sigma}_e^2}{n_i}  }$.
	\item
	For $l=1,\ldots,L$:
	\begin{enumerate}
		\item
		Generate ${v}_i^{(l)} \overset{iid}{\sim} N\big(0,\hat{\sigma}_u^2(1-\hat{\gamma_i})\big)$ and  $e_{ij}^{(l)}\overset{iid}{\sim}N\big(0,\hat{\sigma}_e^2\big)$ and obtain a pseudo-population of the target variable by: $${{y}_{ij}}{}^{(l)}=\mathbf{x}_{ij}^{T}\hat{{\boldsymbol{\beta}}}+\hat u_{i}+v_{i}^{(l)}+e_{ij}^{(l)},$$ where the predicted random effect $\hat u_{i}$ is defined as $\hat u_{i}=E(u_i|\mathbf{y}_{is})$.
		\item
		Calculate the indicator of interest $I_i^{(l)}$ in each area.
	\end{enumerate}
	\item
	Finally, take the mean over the $L$ Monte Carlo runs in each area to obtain a point estimate of the indicator of interest: $$\hat{I}_{i}^{EBP} = \frac{1}{L} \sum\limits_{l=1}^{L}{I}_i^{(l)}.$$
\end{enumerate}



\subsubsection{Empirical Plug-in Predictor (EPP) for binary variables}
    
     In cases when the variable of interest in small areas is binary, e.g. $y_{ij} =0$ or $1$, the target estimation in each domain is the proportion $\bar{Y}_i = \pi_i = \frac{1}{N_i} \sum_{j=1}^{N_i} y_{ij} $, and $\pi_{ij}$ is the probability that a specific unit $j$ in the domain $i$ obtains the value 1. Traditionally, it is assumed that the $\pi_{ij}$ follows a GLMM with a logistic link function defined as: 

    \begin{equation}\label{eq:Plugin1}
        \text{logit}(\pi_{ij}) = \text{log} \Big (   \frac{\pi_{ij}} {1-\pi_{ij}} \Big) = \eta_{ij} = x^T_{ij}\beta + u_i
    \end{equation}
     
    with $j=1, \dots, N_i$, $i=1, \dots, D$, $\beta$ is a vector of fixed effect parameter, and $u_i$ the random area-specific effect for the domain $i$ with $u_i \sim N(0,\sigma^2)$. $u_i$ are assumed independent and $y_{ij}|u_i \sim \text{Binomial}(1, \pi_{ij})$ with $E(y_{ij}|u_i) = \pi_{ij}$ and $Var(y_{ij}|u_i) = \sigma_{ij}=\pi_{ij}(1-\pi_{ij})$. Under model \ref{eq:Plugin1}, $E(y_{ij}|u_i) = \pi_{ij} = \frac{\text{exp}(\eta_{ij})} {{1+\text{exp}(\eta_{ij})}}$. 
    
    As explained in Jiang (2003), for this kind of data the Empirical Best Predictor must be computed by numerical approximation means and for this reason, the Empirical Plug-in Predictor EPP is popularly chosen as a more feasible alternative (Chandra et al (2012)). The EPP for small area proportion in area $i$ is: 
    
    \begin{equation}
        \hat{P}_i = \frac{1}{N_i} \big( \sum_{j \in s_i} y_{ij}+\sum_{j \in r_i} \hat{\pi}_{ij} \big),
    \end{equation}
    
    where $\hat{\pi}_{ij}= \hat{E}(y_{ij}|u_i)= \frac{\text{exp}(\bold{x}^{T}_{ij}\hat{\beta}+\hat{u}_i)} {{1+\text{exp}(\bold{x}^{T}_{ij}\hat{\beta}+\hat{u}_i))}}$. Here $\hat{\beta}$ and $\hat{u}_i$ are the estimate of fixed effects parameter and prediction of random effects parameter, respectively under model \ref{eq:Plugin1}.
    
    The EPP estimator is based on the observed value of study variable for each unit in the sample and predicted values of non-sampled units under model \ref{eq:Plugin1}. This estimator requires unit-level values of auxiliary variables for the population units. 
     

\subsection{Metodología empleada}

-Describir cómo hacemos el proceso para unir las estimaciones al ``censo". 

-Describir cómo se computa el índice. 




\subsection{Measure of uncertainty}

\textcolor{red} {Decidir qué vamos a hacer con el MSE??}
\begin{itemize}
    \item \textcolor{red} {Bootstrap para obtener cada MSE por separado?, reportamos MSE independientemente?}
    \item \textcolor{red} {Bootstrap para obtener cada MSE conjunto? ``sumamos" simplemente o buscamos otra alternativa? }
\end{itemize}



\section{Validation: Design-based simulation study}

\textcolor{red} {Queremos validar el MSE?}
\


\begin{itemize}
    \item In this section we illustrate some of the aspects of composite indicators in a SAE context evaluation. In particular, combining the results of model prediction we described
in sections bellow, we present results for the estimation of the MPI. We discuss how the design-based simulation results can guide the production of the
final set of MPI estimates in a SAE context.
Analysis with the original sam. And we ple. %Table 3.2 presents summaries over municipalities of point, root MSE (RMSE) and CV estimatescomputed using the original data supplied to us by CEPAL and estimated MSEs under the assumed models. 
\item Evaluation of the performance of the proposed methodology in the context of a real population and realistic sampling methods.  
    \item Illustrating aspects of SAE evaluation using the GEIH data %ENIGH
\item
\end{itemize}

\section{Results}


\subsection{Employment-related income estimation}

\textcolor{red}{Iniciar con descripción del modelo. No hay seleccion de variables, ambos modelos usan las mismas. Por esto podríamos decidir si al iniciar la sección de resultados, queremos presentar las variables utilizadas, los tamaños de muestra, población, dominios en /fuera de muestra, etc.?}


An EBP model to obtain employment-related income information was conducted. Income per capita of each individual as dependent variable and the socio-economic covariates described in Table \ref{tab:variables} as predictors. . 

\begin{table}[ht!] 
\begin{center}
  \begin{threeparttable} 
	\centering
\small{
	\caption{Covariables included in the EBP model to obtain employment-related income estimates} \label{tab:variables}
		\begin{tabular}{ll}
		\hline
		Category& Variable \\
		\hline
				 Geographical & 1. Department \\
		 	Socio-demographic 	& 2. Age \\
				& 3. Sex \\
			Education	& 4. Highest degree of education completed   \\
			Employment	& 5. Labor condition \\
			Household 	& 6. Proportion of employees in the household \\
			 conditions& 7. Equivalized size of the household \\
			 	& 8. Overcrowding \\
				& 9. At least one member without health insurance \\		
				& 10. Quantity of economically dependent members\\
				Housing & 11. Poor condition of the floor or ceiling \\
				& 12. Use of internet\\
				 & 13. No garbage disposal system \\
				 & 14. No exclusive toilet for the household \\
			\hline
	\end{tabular}
	}
  \end{threeparttable}
  \end{center}
\end{table}


 To select the model, several transformations were considered to achieve normality in the error terms, but the final version applies a Box-Cox transformation (\citealt{box1964}) with an optimal lambda () using the Restricted maximum likelihood (REML) approach. For this example, normality for both, the unit level and the random effects, is assumed.  Also, the marginal $R^2 =  $ and the conditional $R^2=$ were observed.


\

\begin{table}[ht] 
\small{
	\caption{Summary statistics for sample and population sizes}
	}
	\centering
		\begin{tabular}{lrrrrrr}
		\hline
         &Min &1st Q &Median& Mean & 3rd Q & Max \\ 
		\hline
		  Sample domains &  \multirow{2}{*}{} & \multirow{2}{*}{} & \multirow{2}{*}{}& \multirow{2}{*}{} & \multirow{2}{*}{} & \multirow{2}{*}{}   \\
		  (In-sample: \%) & &&&&&\\
		  Population domains &    &  & &  &  & \\
			\hline
	\end{tabular} 
		\label{tab:sizes}
\end{table}

\begin{itemize}
\item \textcolor{red}{Chequear diagnósticos de los residuos}
\item  \textcolor{red}{QQ plots}
\item  \textcolor{red}{Indicadores básicos: R2, ICC}
\item \textcolor{red}{Comparamos estos resultados con distintas transformaciones o no?}
\end{itemize}


\subsection{Social security and pension estimation}

\begin{itemize}
    \item  \textcolor{red}{Decidir si describimos las variables del modelo otra vez aquí o una subsección inicial, al comienzo de la sección de resultados}
    \item  \textcolor{red}{Chequear residuos para los efectos aleatorios: QQplot y Shapiro test}
    \item \textcolor{red}{análisis de clasificación con los datos de encuesta: sensibilidad, especificidad, exactitud}
\end{itemize}
\end{itemize}


\subsection{Multidimensional deprivation index}


\begin{itemize}
    \item General incidence of poverty by: sex, age group, area.
    \item Description by indicator /domains
    \item Provide information with coefficients of variation 
\end{itemize}






\section{Concluding remarks and further research}


\begin{itemize}
    \item Time-gap between census and surveys.
    \item Possible (co-)relations between indicators and dimensions. 
    \item Provide a formal MSE. 
    \item Generalise the methodology for a) all Latin American countries, and if possible b) composite indicators. 
\end{itemize}

\textbf{Acknowledgments}


\newpage

\bibliography{referencias}
\end{document}
